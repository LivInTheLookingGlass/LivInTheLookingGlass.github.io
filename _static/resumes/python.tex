\documentclass[10pt, letterpaper]{article}

% Packages:
\usepackage[
    ignoreheadfoot, % set margins without considering header and footer
    top=1.75 cm, % seperation between body and page edge from the top
    bottom=1.5 cm, % seperation between body and page edge from the bottom
    left=1.75 cm, % seperation between body and page edge from the left
    right=1.75 cm, % seperation between body and page edge from the right
    footskip=1.0 cm, % seperation between body and footer
    % showframe % for debugging 
]{geometry} % for adjusting page geometry
\usepackage{titlesec} % for customizing section titles
\usepackage{tabularx} % for making tables with fixed width columns
\usepackage{array} % tabularx requires this
\usepackage[dvipsnames]{xcolor} % for coloring text
\definecolor{primaryColor}{RGB}{0, 0, 0} % define primary color
\usepackage{enumitem} % for customizing lists
\usepackage{fontawesome5} % for using icons
\usepackage{amsmath} % for math
\usepackage[
    pdftitle={Olivia Appleton-Crocker's Resume},
    pdfauthor={Olivia Appleton-Crocker},
    pdfcreator={LaTeX with RenderCV},
    colorlinks=true,
    urlcolor=primaryColor
]{hyperref} % for links, metadata and bookmarks
\usepackage[pscoord]{eso-pic} % for floating text on the page
\usepackage{calc} % for calculating lengths
\usepackage{bookmark} % for bookmarks
\usepackage{lastpage} % for getting the total number of pages
\usepackage{changepage} % for one column entries (adjustwidth environment)
\usepackage{paracol} % for two and three column entries
\usepackage{ifthen} % for conditional statements
\usepackage{needspace} % for avoiding page brake right after the section title
\usepackage{iftex} % check if engine is pdflatex, xetex or luatex

% Ensure that generate pdf is machine readable/ATS parsable:
\ifPDFTeX
    \input{glyphtounicode}
    \pdfgentounicode=1
    \usepackage[T1]{fontenc}
    \usepackage[utf8]{inputenc}
    \usepackage{lmodern}
\fi

\usepackage{charter}

% Some settings:
\raggedright
\AtBeginEnvironment{adjustwidth}{\partopsep0pt} % remove space before adjustwidth environment
\pagestyle{empty} % no header or footer
\setcounter{secnumdepth}{0} % no section numbering
\setlength{\parindent}{0pt} % no indentation
\setlength{\topskip}{0pt} % no top skip
\setlength{\columnsep}{0.15cm} % set column seperation
\pagenumbering{gobble} % no page numbering

\titleformat{\section}{\needspace{4\baselineskip}\bfseries\large}{}{0pt}{}[\vspace{1pt}\titlerule]

\titlespacing{\section}{
    % left space:
    -1pt
}{
    % top space:
    0.3 cm
}{
    % bottom space:
    0.2 cm
} % section title spacing

\renewcommand\labelitemi{$\vcenter{\hbox{\small$\bullet$}}$} % custom bullet points
\newenvironment{highlights}{
    \begin{itemize}[
        topsep=0.10 cm,
        parsep=0.10 cm,
        partopsep=0pt,
        itemsep=0pt,
        leftmargin=0 cm + 10pt
    ]
}{
    \end{itemize}
} % new environment for highlights


\newenvironment{highlightsforbulletentries}{
    \begin{itemize}[
        topsep=0.10 cm,
        parsep=0.10 cm,
        partopsep=0pt,
        itemsep=0pt,
        leftmargin=10pt
    ]
}{
    \end{itemize}
} % new environment for highlights for bullet entries

\newenvironment{onecolentry}{
    \begin{adjustwidth}{
        0 cm + 0.00001 cm
    }{
        0 cm + 0.00001 cm
    }
}{
    \end{adjustwidth}
} % new environment for one column entries

\newenvironment{twocolentry}[2][]{
    \onecolentry
    \def\secondColumn{#2}
    \setcolumnwidth{\fill, 4.5 cm}
    \begin{paracol}{2}
}{
    \switchcolumn \raggedleft \secondColumn
    \end{paracol}
    \endonecolentry
} % new environment for two column entries

\newenvironment{threecolentry}[3][]{
    \onecolentry
    \def\thirdColumn{#3}
    \setcolumnwidth{, \fill, 4.5 cm}
    \begin{paracol}{3}
    {\raggedright #2} \switchcolumn
}{
    \switchcolumn \raggedleft \thirdColumn
    \end{paracol}
    \endonecolentry
} % new environment for three column entries

\newenvironment{header}{
    \setlength{\topsep}{0pt}\par\kern\topsep\centering\linespread{1.5}
}{
    \par\kern\topsep
} % new environment for the header

\newcommand{\placelastupdatedtext}{% \placetextbox{<horizontal pos>}{<vertical pos>}{<stuff>}
  \AddToShipoutPictureFG*{% Add <stuff> to current page foreground
    \put(
        \LenToUnit{\paperwidth-2 cm-0 cm+0.05cm},
        \LenToUnit{\paperheight-1.0 cm}
    ){\vtop{{\null}\makebox[0pt][c]{
        \small\color{gray}\textit{Last updated in January 2025}\hspace{\widthof{Last updated in January 2025}}
    }}}%
  }%
}%

% save the original href command in a new command:
\let\hrefWithoutArrow\href

% new command for external links:


\begin{document}
    \newcommand{\AND}{\unskip
        \cleaders\copy\ANDbox\hskip\wd\ANDbox
        \ignorespaces
    }
    \newsavebox\ANDbox
    \sbox\ANDbox{$|$}

    \begin{header}
        \fontsize{25 pt}{25 pt}\selectfont Olivia Appleton-Crocker

        \normalsize
        \mbox{Chicago, IL}%
        \kern 5.0 pt%
        \AND%
        \kern 5.0 pt%
        \mbox{\hrefWithoutArrow{mailto:liv@oliviaappleton.com}{liv@oliviaappleton.com}}%
        \kern 5.0 pt%
        \AND%
        \kern 5.0 pt%
        \mbox{\hrefWithoutArrow{tel:+1-906-361-9876}{+1-906-361-9876}}%
        \kern 5.0 pt%
        \AND%
        \kern 5.0 pt%
        \mbox{\hrefWithoutArrow{https://oliviaappleton.com/}{oliviaappleton.com}}%
        \kern 5.0 pt%
        \AND%
        \kern 5.0 pt%
        \mbox{\hrefWithoutArrow{https://linkedin.com/in/olivia-kay-appleton}{linkedin.com/in/olivia-kay-appleton}}%
        \kern 5.0 pt%
        \AND%
        \kern 5.0 pt%
        \mbox{\hrefWithoutArrow{https://github.com/LivInTheLookingGlass}{github.com/LivInTheLookingGlass}}%
    \end{header}

    \vspace{-10 pt}

    \section{Education}

        
        \begin{twocolentry}{
            Jan. 2019 - Dec. 2022
        }
            \textbf{Michigan State University}, Master's in Computer Science \& Engineering
        \end{twocolentry}

        \vspace{0.10 cm}
        \begin{onecolentry}
            \begin{highlights}
                \item GPA: 3.85/4.0
                \item \textbf{Coursework:} Discrete Logic, Distributed Systems, Foundations of Computing, Machine Learning
            \end{highlights}
        \end{onecolentry}

        \vspace{0.10 cm}
        
        \begin{twocolentry}{
            Sep. 2013 - Dec. 2018
        }
            \textbf{Northern Michigan University}, BS in Computer Science\end{twocolentry}

        \vspace{0.10 cm}
        \begin{onecolentry}
            \begin{highlights}
                \item GPA: 3.84/4.0 (Magna cum laude)
                \item \textbf{Coursework:} Data Structures, Microcomputer Architecture, Networking, Object-Oriented Design, Operating Systems, Principles of Programming Languages
            \end{highlights}
        \end{onecolentry}

    \vspace{-5pt}

    \section{Experience}
        
        \begin{twocolentry}{
            May 2024 – Present
        }
            \textbf{Data Science Fellow}, TMW Center for Early Learning + Public Health -- Chicago, IL
        \end{twocolentry}

        \vspace{0.10 cm}
        \begin{onecolentry}
            \begin{highlights}
                \item Raising backend code ($\sim$19k lines) coverage by 25+ percentage points
                \item Wrote code in C\#, TypeScript, JavaScript, and Python
                \item Assisted in integrating two programming teams
            \end{highlights}
        \end{onecolentry}

        \vspace{0.1 cm}

        \begin{twocolentry}{
            Jan. 2020 - Feb. 2023
        }
            \textbf{Teaching and Research Assistant}, Michigan State University -- East Lansing, MI
        \end{twocolentry}

        \vspace{0.10 cm}
        \begin{onecolentry}
            \begin{highlights}
                \item Published 2 papers, where the relevant code was written in Python
                \item Assisted teaching classes, including one where we implemented SQLite from scratch in Python 3
                \item Consistent high reviews from students
            \end{highlights}
        \end{onecolentry}

        \vspace{0.1 cm}

        \begin{twocolentry}{
            Jan. 2018 - Dec. 2019 \\
            May 2015 - Sep. 2016
        }
            \textbf{Product Development Engineer (Various Titles)}, Intel (NSG) -- Folsom, CA
        \end{twocolentry}

        \vspace{-0.35 cm}
        \begin{onecolentry}
            \begin{highlights}
                \item Coordinated a small team of programmers (3-5 people at any given time)
                \item Helped design a testing protocol for NVMe's Power Loss Notification
                \item Influenced changes to the NVMe specification
                \item Rewrote internal tools to streamline and comply with Python 3
                \item Built software models of various pre-market products
            \end{highlights}
        \end{onecolentry}

    \vspace{-5pt}

    \section{Publications}

        \begin{samepage}
            \begin{twocolentry}{
                Jan. 2022
            }
                \textbf{Achieving Causality with Physical Clocks}
            \end{twocolentry}

            \vspace{0.10 cm}
            
            \begin{onecolentry}
                \mbox{Sandeep S Kulkarni}, \mbox{\textbf{\textit{Olivia Appleton-Crocker}}}, \mbox{Duong Nguyen}

                \vspace{0.10 cm}
                
        \href{https://doi.org/10.1145/3491003.3491009}{10.1145/3491003.3491009}
        \end{onecolentry}
        \end{samepage}

        \begin{samepage}
            \begin{twocolentry}{
                July 2020
            }
                \textbf{Efficient Two-Layered Monitor for Partially Synchronous Distributed Systems (Technical Report)}
            \end{twocolentry}

            \vspace{0.10 cm}
            
            \begin{onecolentry}
                \mbox{Vidhya Tekken Valapil}, \mbox{Sandeep S Kulkarni}, \mbox{Eric Torng}, \mbox{\textbf{\textit{Olivia Appleton-Crocker}}}

                \vspace{0.10 cm}
                
        \href{https://doi.org/10.48550/arXiv.2007.13030}{10.48550/arXiv.2007.13030}
        \end{onecolentry}
        \end{samepage}

    \vspace{-5pt}
    
    \section{Projects}
        
        \begin{twocolentry}{
            \href{https://github.com/python/cpython}{github.com/python/cpython}
        }
            \textbf{CPython}
        \end{twocolentry}

        \begin{onecolentry}
            \begin{highlights}
                \item Added support for the UDPLite network protocol
                \item Tools Used: C, Python, Sphinx
            \end{highlights}
        \end{onecolentry}

        \vspace{0.1 cm}

        \begin{twocolentry}{
            \href{https://euler.oliviaappleton.com}{euler.oliviaappleton.com}
        }
            \textbf{Showcase: Project Euler Solutions}
        \end{twocolentry}

        \begin{onecolentry}
            \begin{highlights}
                \item Solutions in 9 different languages to various math programming puzzles
                \item Tools Used: C, C+\!+, C\#, CI/CD, Fortran, Java, JavaScript, Lua, Makefile, Python, Rust, Sphinx, WebAssembly
            \end{highlights}
        \end{onecolentry}

    \vspace{-5pt}

    \section{Technologies}

        \begin{onecolentry}
            \textbf{Languages:} Python, C/C+\!+, C\#, Rust, JavaScript, SQL, Java, Bash, Fortran, Lua, SmallTalk
        \end{onecolentry}

        \begin{onecolentry}
            \textbf{Technologies:} Cypress, .NET, Makefile, CI/CD, Github Actions
        \end{onecolentry}

\end{document}
