\documentclass[10pt, letterpaper]{article}

% Packages:
\usepackage[
	ignoreheadfoot, % set margins without considering header and footer
	top=1 cm, % seperation between body and page edge from the top
	bottom=1 cm, % seperation between body and page edge from the bottom
	left=1.25 cm, % seperation between body and page edge from the left
	right=1.25 cm, % seperation between body and page edge from the right
	footskip=0 cm, % seperation between body and footer
	% showframe % for debugging
]{geometry} % for adjusting page geometry
\usepackage{titlesec} % for customizing section titles
\usepackage{tabularx} % for making tables with fixed width columns
\usepackage{array} % tabularx requires this
\usepackage[dvipsnames]{xcolor} % for coloring text
\definecolor{primaryColor}{RGB}{0, 0, 0} % define primary color
\usepackage{enumitem} % for customizing lists
\usepackage{fontawesome5} % for using icons
\usepackage{amsmath} % for math
\usepackage[
	pdftitle={Olivia Appleton-Crocker's Resume},
	pdfauthor={Olivia Appleton-Crocker},
	pdfcreator={LaTeX with RenderCV},
	colorlinks=true,
	urlcolor=primaryColor
]{hyperref} % for links, metadata and bookmarks
\usepackage[pscoord]{eso-pic} % for floating text on the page
\usepackage{calc} % for calculating lengths
\usepackage{bookmark} % for bookmarks
\usepackage{lastpage} % for getting the total number of pages
\usepackage{changepage} % for one column entries (adjustwidth environment)
\usepackage{paracol} % for two and three column entries
\usepackage{ifthen} % for conditional statements
\usepackage{needspace} % for avoiding page brake right after the section title
\usepackage{iftex} % check if engine is pdflatex, xetex or luatex

% Ensure that generate pdf is machine readable/ATS parsable:
\ifPDFTeX
\input{glyphtounicode}
\pdfgentounicode=1
\usepackage[T1]{fontenc}
\usepackage[utf8]{inputenc}
\usepackage{lmodern}
\fi

\usepackage{charter}

% Some settings:
\raggedright
\AtBeginEnvironment{adjustwidth}{\partopsep0pt} % remove space before adjustwidth environment
\pagestyle{empty} % no header or footer
\setcounter{secnumdepth}{0} % no section numbering
\setlength{\parindent}{0pt} % no indentation
\setlength{\topskip}{0pt} % no top skip
\setlength{\columnsep}{0.15cm} % set column seperation
\pagenumbering{gobble} % no page numbering

\titleformat{\section}{\needspace{4\baselineskip}\bfseries\large}{}{0pt}{}[
\vspace{1pt}
\titlerule]

\titlespacing{\section}{
% left space:
-1pt }{
% top space:
0.3 cm }{
% bottom space:
0.2 cm } % section title spacing

\renewcommand{\labelitemi}{$\vcenter{\hbox{\small$\bullet$}}$} % custom bullet points
\newenvironment{highlights}{ \begin{itemize}[ topsep=0.10 cm, parsep=0.10 cm, partopsep=0pt,
itemsep=0pt, leftmargin=0 cm + 10pt ] }{ \end{itemize} } % new environment for highlights

\newenvironment{highlightsforbulletentries}{ \begin{itemize}[ topsep=0.10 cm,
parsep=0.10 cm, partopsep=0pt, itemsep=0pt, leftmargin=10pt ] }{ \end{itemize} } % new environment for highlights for bullet entries

\newenvironment{onecolentry}{ \begin{adjustwidth}{ 0 cm + 0.00001 cm }{ 0 cm + 0.00001 cm }
}{ \end{adjustwidth} } % new environment for one column entries

\newenvironment{twocolentry}[2][]{ \onecolentry \def\secondColumn{#2} \setcolumnwidth{\fill, 4.5 cm}
\begin{paracol}{2} }{ \switchcolumn \raggedleft \secondColumn  \end{paracol}
\endonecolentry } % new environment for two column entries

\newenvironment{threecolentry}[3][]{ \onecolentry \def\thirdColumn{#3} \setcolumnwidth{, \fill, 4.5 cm}
\begin{paracol}{3} {\raggedright #2} \switchcolumn }{ \switchcolumn \raggedleft \thirdColumn
\end{paracol} \endonecolentry } % new environment for three column entries

\newenvironment{header}{
\setlength{\topsep}{0pt}
\par\kern\topsep
\centering
\linespread{1.5} }{ \par\kern\topsep } % new environment for the header

\newcommand{\placelastupdatedtext}{% \placetextbox{<horizontal pos>}{<vertical pos>}{<stuff>}
\AddToShipoutPictureFG*{% Add <stuff> to current page foreground
\put( \LenToUnit{\paperwidth-2 cm-0 cm+0.05cm}, \LenToUnit{\paperheight-1.0 cm} ){\vtop{{\null}\makebox[0pt][c]{ \small\color{gray}\textit{Last updated in January 2025}\hspace{\widthof{Last updated in January 2025}} }}}%
}%
}%

% save the original href command in a new command:
\let\hrefWithoutArrow\href

% new command for external links:

\begin{document}
	\newcommand{\AND}{\unskip \cleaders\copy\ANDbox\hskip\wd\ANDbox \ignorespaces }
	\newsavebox{\ANDbox}
	\sbox{\ANDbox}{$|$}

	\begin{header}
		\fontsize{25 pt}{25 pt}\selectfont Olivia Appleton-Crocker

		\normalsize \mbox{Chicago, IL}%
		\kern 3.0 pt%
		\AND%
		\kern 3.0 pt%
		\mbox{\hrefWithoutArrow{tel:+1-906-361-9876}{+\!1-906-361-9876}}%
		\kern 3.0 pt%
		\AND%
		\kern 3.0 pt%
		\mbox{\hrefWithoutArrow{https://oliviaappleton.com/}{oliviaappleton.com}}%
		\kern 3.0 pt%
		\AND%
		\kern 3.0 pt%
		\mbox{\hrefWithoutArrow{mailto:liv@oliviaappleton.com}{liv@oliviaappleton.com}}%
		\kern 3.0 pt%
		\AND%
		\kern 3.0 pt%
		\mbox{\hrefWithoutArrow{https://github.com/LivInTheLookingGlass}{github.com/LivInTheLookingGlass}}%
	\end{header}

	\vspace{-7 pt}
	\vspace{0.6 cm}

	\section{Education}

	\begin{twocolentry}
		{ Jan. 2020 - Dec. 2022 } \textbf{Michigan State University}, Master's in Computer
		Science \& Engineering
	\end{twocolentry}

	\vspace{0.10 cm}
	\begin{onecolentry}
		\begin{highlights}
			\item GPA: 3.85/4.0 \item {\textbf{Coursework:} Discrete Logic, Distributed Systems, Foundations of Computing, Machine Learning,\\
			Algorithmic Graph Theory, Parallel Computing}
		\end{highlights}
	\end{onecolentry}

	\vspace{0.20 cm}

	\begin{twocolentry}
		{ Sep. 2013 - Dec. 2018 } \textbf{Northern Michigan University}, BS in Computer
		Science
	\end{twocolentry}

	\vspace{0.10 cm}
	\begin{onecolentry}
		\begin{highlights}
			\item GPA: 3.84/4.0 (Magna cum laude)
			\item \textbf{Coursework:} {Data Structures, Microcomputer Architecture, Networking, Object-Oriented Design, \\
			Operating Systems, Principles of Programming Languages, Algorithm Design \& Analysis}
		\end{highlights}
	\end{onecolentry}


	\vspace{0.6 cm}

	\section{Experience}

	\begin{twocolentry}
		{ May 2024 – Present } \textbf{Data Science Fellow}, TMW Center for Early Learning
		+ Public Health -- Chicago, IL
	\end{twocolentry}

	\vspace{0.10 cm}
	\begin{onecolentry}
		\begin{highlights}
			\item Raising backend code ($\sim$19k lines) coverage by 25+ percentage points
			\item Wrote code in C\#, TypeScript, JavaScript, and Python
			\item Triaged communications problems between custom hardware and the app/servers that maintained them
			\item Assisted in integrating two programming teams
		\end{highlights}
	\end{onecolentry}

	\vspace{0.2 cm}

	\begin{twocolentry}
		{ Jan. 2020 - Feb. 2023 } \textbf{Teaching and Research Assistant}, Michigan
		State University -- East Lansing, MI
	\end{twocolentry}

	\vspace{0.10 cm}
	\begin{onecolentry}
		\begin{highlights}
			\item Published 2 papers, where the relevant code was written in Python
			\item Assisted teaching classes, including one where we implemented SQLite from scratch in Python 3
			\item Provided numerous tutoring sessions in both math and programming
			\item Consistent high reviews from students
		\end{highlights}
	\end{onecolentry}

	\vspace{0.2 cm}

	\begin{twocolentry}
		{ Jan. 2018 - Dec. 2019 \\ May 2015 - Sep. 2016 } \textbf{Product
		Development Engineer (Various Titles)}, Intel (NSG) -- Folsom, CA
	\end{twocolentry}

	\vspace{-0.35 cm}
	\begin{onecolentry}
		\begin{highlights}
			\item Coordinated a small team of programmers (3-5 people at any given
			time) \item Helped design a testing protocol for NVMe's Power Loss Notification
			\item Influenced changes to the NVMe specification \item Rewrote internal tools
			to streamline and comply with Python 3
			\item Built software models of various pre-market products
		\end{highlights}
	\end{onecolentry}

	\vspace{0.6 cm}

	\section{Publications}

	\begin{samepage}
		\begin{twocolentry}
			{ Jan. 2022 } \textbf{Achieving Causality with Physical Clocks}
		\end{twocolentry}

		\vspace{0.10 cm}

		\begin{onecolentry}
			\mbox{Sandeep S Kulkarni},
			\mbox{\textbf{\textit{Olivia Appleton-Crocker}}}, \mbox{Duong Nguyen}

			\vspace{0.10 cm}

			\href{https://doi.org/10.1145/3491003.3491009}{10.1145/3491003.3491009} \vspace{0.1 cm}

			This paper presented a novel way to encode causality information in the least-significant bits of a timestamp. Computers that recognize this encoding can use it to order events more certainly, while computers that do not can safely treat it as a standard NTP timestamp.
		\end{onecolentry}
	\end{samepage}

	\vspace{0.2 cm}

	\begin{samepage}
		\begin{twocolentry}
			{ July 2020 } \textbf{Efficient Two-Layered Monitor for Partially
			Synchronous Distributed Systems}
		\end{twocolentry}

		\vspace{0.10 cm}

		\begin{onecolentry}
			\mbox{Vidhya Tekken Valapil}, \mbox{Sandeep S Kulkarni}, \mbox{Eric Torng},
			\mbox{\textbf{\textit{Olivia Appleton-Crocker}}}

			\vspace{0.10 cm}

			\href{https://doi.org/10.48550/arXiv.2007.13030}{10.48550/arXiv.2007.13030} \vspace{0.1 cm}

			This paper presents a novel way to monitor distributed systems with much lower cost than the previous standard of using vector clocks. Two layers are used: one which is cheap but imprecise, and one that is precise but more costly. In tandem they reduce monitoring costs by at least 85\%.
		\end{onecolentry}
	\end{samepage}

	\vfill

	\begin{center}
		\footnotesize (More on next page) \normalsize
	\end{center}

	\newpage

	\section{Projects}

	\begin{samepage}
		\begin{twocolentry}
			{ \href{https://github.com/python/cpython}{github.com/python/cpython} } \textbf{CPython}
		\end{twocolentry}
	
		\begin{onecolentry}
			\begin{highlights}
				\item Added support for the UDPLite network protocol
				\item Tools Used: C, Python, Sphinx, UnitTest
			\end{highlights}
		\end{onecolentry}
	\end{samepage}

	\vspace{0.2 cm}

	\begin{samepage}
		\begin{twocolentry}
			{ \href{https://euler.oliviaappleton.com}{euler.oliviaappleton.com} } \textbf{Showcase:
			Project Euler Solutions}
		\end{twocolentry}
	
		\begin{onecolentry}
			\begin{highlights}
				\item Solutions in 9 different languages to various math programming puzzles, including extensive prime number toolkit
				\item Tools Used: C, C+\!+, C\#, CI/CD, Fortran, Java, JavaScript, Lua,
				Makefile, Python, Rust, Sphinx, WebAssembly
			\end{highlights}
		\end{onecolentry}
	\end{samepage}

	\vspace{0.2 cm}

	\begin{samepage}
		\begin{twocolentry}
			{ \href{https://github.com/LivInTheLookingGlass/overpassify}{\hspace{-3 cm} \mbox{github.com/LivInTheLookingGlass/overpassify}} } \textbf{Overpassify}
		\end{twocolentry}
	
		\begin{onecolentry}
			\begin{highlights}
				\item A transpiler that turns Python code into OpenStreetMap's OverpassQL query language
				\item This is useful because OverpassQL is often difficult to read
				\item Tools Used: Makefile, OpenStreetMap, OverpassQL, Python
			\end{highlights}
		\end{onecolentry}
	\end{samepage}

	\vspace{0.2 cm}

	\begin{samepage}
		\begin{twocolentry}
			{\hspace{-3 cm} \href{https://github.com/p2p-today/p2p-project}{ \mbox{github.com/p2p-today/p2p-project}} } \textbf{Undergrad Dissertatation Project}
		\end{twocolentry}
	
		\begin{onecolentry}
			\begin{highlights}
				\item A multi-language, interoperable, peer-to-peer network framework
				\item Implemented several network types, including a mesh network \& two types of distributed hash table
				\item Full implementation was done in Python, JavaScript. Message serialization demonstrated in C, C+\!+, Java, \& SmallTalk
				\item Tools Used: Makefile, Python, PyTest, JavaScript, Babel, Mocha, WebPack, Node.JS, C, C+\!+, Java, SmallTalk
			\end{highlights}
		\end{onecolentry}
	\end{samepage}

	\vspace{0.2 cm}

	\begin{samepage}
		\begin{twocolentry}
			{\hspace{-3 cm} \href{https://github.com/LivInTheLookingGlass/job-splitter}{ \mbox{github.com/LivInTheLookingGlass/job-splitter}} } \textbf{Research Job Splitter}
		\end{twocolentry}
	
		\begin{onecolentry}
			\begin{highlights}
				\item This tool would programmatically split the simulation tasks I needed to run across many nodes
				\item The nodes did not need to communicate with each other
				\item Packaged the tasks into a single executable
				\item Tools Used: Python, Makefile, PyInstaller, Multiprocessing
			\end{highlights}
		\end{onecolentry}
	\end{samepage}

	\vspace{0.2 cm}

	\begin{samepage}
		\begin{twocolentry}
			{\hspace{-7 cm} \href{https://github.com/LivInTheLookingGlass/ManifoldMarketManager}{\mbox{github.com/LivInTheLookingGlass/ManifoldMarketManager}} } \textbf{ManifoldMarketManager}
		\end{twocolentry}
	
		\begin{onecolentry}
			\begin{highlights}
				\item Manifold Markets is a play-money betting site about real-world events
				\item Before they deployed a new API, this bot would make and manage markets
				\item It implemented a large suite of rules, including logical combinations of other rules, or artificially leveraged odds
				\item It could also query a human administrator for a resolution over Telegram
				\item Required me to implement hooks for all of the relevant API calls
				\item Tools Used: Python, REST API, Telegram, SQLite, PyTest
			\end{highlights}
		\end{onecolentry}
	\end{samepage}

	\vspace{0.2 cm}

	\begin{samepage}
		\begin{twocolentry}
			{\hspace{-3 cm} \href{https://openstreetmap.org/LivInTheLookingGlass}{\mbox{openstreetmap.org/LivInTheLookingGlass}} } \textbf{OpenStreetMap Contributions}
		\end{twocolentry}
	
		\begin{onecolentry}
			\begin{highlights}
				\item OpenStreetMap is an open geographic database, like Wikipedia for maps
				\item It is used frequently by emergency services in disasters
				\item I organized two public map-a-thons to work in the aftermath of two such global disasters
			\end{highlights}
		\end{onecolentry}
	\end{samepage}

	\vspace{0.6 cm}

	\section{Technologies}

	\begin{onecolentry}
		\textbf{Languages:} Python, C/C+\!+, C\#, Rust, JavaScript, SQL, Java, Bash,
		Fortran, Lua, SmallTalk
	\end{onecolentry}

	\begin{onecolentry}
		\textbf{Technologies:} CI/CD, Cypress, Github Actions, Makefile, Mocha, Moq, .NET, PyTest, UnitTest
	\end{onecolentry}
\end{document}
